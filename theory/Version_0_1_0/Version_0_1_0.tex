\documentclass{article}
\usepackage{amsmath}
\usepackage{amssymb}
\usepackage{bbm}
\usepackage{url}
\usepackage{graphicx}
\usepackage{float}
\title{A One-Period Crowdfunding Model}
\begin{document}
\maketitle
Suppose a startup designs a new product, yet
its limited cash flow or a high fixed cost precludes
it from manufacturing the goods. Hence it initiates
a crowdfunding campaign, by which it collects
money from customers first and then produce the goods.
There are several essential features of a reward-based
crowdfunding campaign:
\begin{itemize}
    \item If one decides to back the project, she pays the
    price immediately. 
    \item The campaign lasts for a time period and the
    production also takes time. Therefore the value
    of the reward is discounted and backing a project
    involves liquidity cost.
    \item The project may fail to reach its goal. In this
    case, no reward is given and the money is refunded.
    Hence the failure of a campaign does not incur
    any monetary cost for the backers.
\end{itemize}

A natural question one may ask is how can a producer st the price
and the goal to raise as much money as possible. To answer
this question, let us first consider a toy model of
one-period crowdfunding.

\section{Model Setting}
There are $N$ consumers and one firm in the market. The firm
has constant marginal cost $c$ and fixed cost $F$ if it decides
to produce goods. A consumer $i$'s valuation of the good $V_i$ is random,
which has probability $q$ to be $\overline{v}$ and $1-q$ to be $\underline{v}$.
We assume $\overline{v} > \underline{v} > c > 0$.

The firm first chooses price $p$ and goal $G$ to initiate a
crowdfunding campaign. Consumers then choose whether to back.
We denote the number of backers by $N_B$.
If a consumer chooses to quit (denoted by $Q$), she incurs no cost.
If she chooses to back (denoted by $B$), she is bound to incurs
a liquidity cost $\underline{L} < \underline{v}$ and
gain $V_i - p$ if $N_B p \geq G$ and $0$ otherwise.
\section{Solving the Model}
We first solve the distribution of demand given a price
$p$ and a goal $G$, and then solve the firm's optimization
problem.

Notice that $N_B p \geq G$ is equivalent to
$N_B \geq ceil(\frac{G}{p}) $. Hence for convenience
we denote $ceil(\frac{G}{p})$ by $\underline{N}$, the
minimum number of backers needed to reach the goal.
Obviously, choosing $G$ is equivalent to choosing $\underline{N}$.

The consumer's decisions given a campaign is essentially a
Bayesian game $ <N,V,(A_i),(T_i),(\tau_i),(P_i),(u_i)> $,
where $V = \times_{i \in N} \{\overline{v},\underline{v}\} $,
$A_i = \{B,Q\}$, $T_i = \{\overline{v},\underline{v}\}$,
$\tau_i(V) = V_i$, $P_i = P_j = P$, the product measure
over $V$ with $P(V_i = \overline{v}) = q$. And the utility
function is:
\begin{equation}
    u_i(a,V) = \begin{cases}
        (V_i - p)\mathbbm{1}_{\sum_{j \in N} \mathbbm{1}_{a_j = B} \geq \underline{N}} - L & \text{if } a_i = B \\
        0 & \text{if } a_i = Q
    \end{cases}
\end{equation}

We first consider the case when $\underline{v} - p > L$, that is,
every consumer would back the project if the reward is guaranteed.

Two equilibria are salient:
$a_i \equiv B \quad \forall i$ or $a_i \equiv Q \quad \forall i$

These results are not interesting and are unlikely to happen in reality:
At least some people back a project and there are consumers complaining
that the uncertainty of the campaign prevents them from
backing it even if they are fond of the good.

Intuition suggests the following separating equilibrium:
\begin{equation}
    a_i = \begin{cases}
        B & \text{if } V_i = \overline{v} \\
        Q & \text{if } V_i = \underline{v}
    \end{cases}
\end{equation}
This means that consumers with low (not enough) WTP
quit due to the uncertainty and those with high WTP
back the project albeit the risk.

The existence of such equilibrium requires:
\begin{equation}
    \begin{cases}
        P(\mathbbm{1}_{\sum_{j \in -i} \mathbbm{1}_{V_j = \overline{v}} \geq \underline{N} - 1})
        (\overline{v}-p) - L \geq 0 \\
        P(\mathbbm{1}_{\sum_{j \in -i} \mathbbm{1}_{V_j = \overline{v}} \geq \underline{N} - 1})
        (\underline{v} - p) - L < 0
    \end{cases}
\end{equation}

The tricky part is
$P(\mathbbm{1}_{\sum_{j \in -i} \mathbbm{1}_{V_j = \overline{v}} \geq \underline{N} - 1})$,
the probability of a random variable following binomial distribution
greater than or equal to some number. Of course we can expand
it as a discrete sum. Yet it has a neater representation:
This probability is equal to the probability of the
$(\underline{N}-1)$th variable in ascending order of $N-1$ independent
random variables $X_1, X_2, \dots X_{N-1}$following uniform distribution on $[0,1]$.
This probability involving ordered statistics can be expressed
by beta distribution:
\begin{align*}
    P(\mathbbm{1}_{\sum_{j \in -i} \mathbbm{1}_{V_j = \overline{v}} \geq \underline{N} - 1})
    &= P(X_{[\underline{N}-1]} \leq q) \\
    &= I_q(\underline{N}-1, N - \underline{N} + 1) \\
    &= \frac{B(q;\underline{N}-1, N - \underline{N} + 1)}{B(\underline{N}-1, N - \underline{N} + 1)}\\
    &= \frac{\Gamma(N)}{\Gamma(\underline{N}-1)\Gamma(N - \underline{N} + 1)}
    \int_0^q t^{\underline{N}-2}(1-t)^{N - \underline{N}}dt
\end{align*}
The function $I_q(\underline{N}-1, N - \underline{N} + 1)$ is called
the regularized incomplete beta function. It is the CDF of
beta distribution.


By solving the above inequalities, we get:
\begin{align}
    \frac{L}{\overline{v} - p} \leq
    I_q(\underline{N}-1, N - \underline{N} + 1) <
    \frac{L}{\underline{v} - p}
\end{align}

Now back to the firm's optimization problem assuming this
condition holds. If the firm is maximizing its expected
profit, it solves:
\begin{align}
    \begin{split}
    \max_{p,\underline{N}} \quad E[\pi] &=
    E[N_B (p-c) \mathbbm{1}_{N_b \geq \underline{N}} -
    F\mathbbm{1}_{N_b \geq \underline{N}}] \\
    &= (p-c)E[N_B \mathbbm{1}_{N_B \geq \underline{N}}] -
    FE[\mathbbm{1}_{N_B \geq \underline{N}}] \\
    \end{split}\\
    s.t. \quad & \frac{L}{\overline{v} - p} \leq
    I_q(\underline{N}-1, N - \underline{N} + 1) <
    \frac{L}{\underline{v} - p}\\
    & where \quad N_B \sim Bin(N,q)
\end{align}

By previous result,
\begin{align*}
    E[\mathbbm{1}_{N_B \geq \underline{N}}] &= P(N_B \geq \underline{N}) \\
    &= I_q(\underline{N}, N - \underline{N} + 1)
\end{align*}
And \begin{align*}
    &E[N_B \mathbbm{1}_{N_B \geq \underline{N}}]\\
    & = \sum_{n_b=\overline{N}}^{N} n_B \binom{N}{n_B} q^{n_B} (1-q)^{N-n_B}\\
    & = Nq \sum_{i=\overline{N}-1}^{N-1} \binom{N-1}{i} q^{i} (1-q)^{N-1-i}\\
    & = Nq I_q(\underline{N}-1, N - \underline{N} + 1)
\end{align*}

So \begin{align}
    E\pi(p,\underline{N}) = Nq(p-c)I_q(\underline{N}-1, N - \underline{N} + 1) - F I_q(\underline{N}, N - \underline{N} + 1)
\end{align}

Since $\underline{N}$ can only take finite values, we can
compute the optimal $E\pi$ given each $\underline{N}$ and
then compare them.

For a fixed $\underline{N}$, $E\pi$ is maximized when
p takes the greatest possible value, which is
$\overline{v} - \frac{L}{I_q(\underline{N}-1, N - \underline{N} + 1)}$.
This yields
\begin{align}
    \begin{split}
    E\pi^*(\underline{N}) 
    &= Nq(\overline{v} - \frac{L}{I_q(\underline{N}-1, N - \underline{N} + 1)} -c)
    I_q(\underline{N}-1, N - \underline{N} + 1) \\ 
    & - F I_q(\underline{N}, N - \underline{N} + 1) \\
    & = Nq(\overline{v} -c)I_q(\underline{N}-1, N - \underline{N} + 1)
    - LNq - F I_q(\underline{N}, N - \underline{N} + 1)
    \end{split}
\end{align}

Take difference with respect to $\underline{N}$ yields:
\begin{align}
    \begin{split}
    &E\pi^*(\underline{N}+1) - E\pi^*(\underline{N})\\
    &=Nq(\overline{v}-c)[I_q(\underline{N}, N - \underline{N}) - I_q(\underline{N}-1, N - \underline{N} + 1)]\\
    & -F[I_q(\underline{N}+1, N - \underline{N}) - I_q(\underline{N}, N - \underline{N} + 1)]
    \end{split}
\end{align}
, for $ 2 \leq \underline{N} \leq N-1 \quad \underline{N} \in \mathbb{N}$
If this difference is a monotonic function of $\underline{N}$, then
the optimal level of threshold is the largest $\overline{N}$
that makes this difference positive.


\section{Initial Result}
However, neither hand computing nor
symbolic computing software managed to give
a closed form of this problem, therefore I
resorted to numeric simulation (\path{simu_version_0_1_0.qmd}).
With a reasonable set of parameters
($N=500$, $q=0.7$, $\overline{v}=100$, $c=50$, $F=1000$, $L=10$),
I got a result that had never been expected:
I had conjectured that
$N*q$ might be the optimal level, since this
is the expected number of high WTP consumers.
However, as Fig \ref{fig:expectation and variance of profit}
indicates, if $\underline{N}$ is
below a certain level, then both the expectation
and the variance of the firm's profit almost
remains unchanged! According to my naive
theory, the project initiator should not
set a goal at all! This result makes sense,
though: For any realization of WTP
distribution, a project with a lower goal
must attract at least as many consumers as
a project with a higher goal.

\begin{figure}[H]
    \label{fig:expectation and variance of profit}
    \centering
    \caption{Expectation and Variance of Profit as $\underline{N}$ increases}
    \includegraphics[width=0.8\textwidth]{fig/expectation_variance_plot.png}
\end{figure}

Then why does this not happen in reality?
The most possible reason is that
the firm not only cares about the
expected profit, but the minium
profit as well, since it needs
backers to cover its fixed cost.
Plus, firms do not know in practice
the total number of consumers and
their preference distribution, thus
may adopt a `maximin'-like strategy
to offset this uncertainty.

It seems that I still have a long way to
go\dots
\end{document}